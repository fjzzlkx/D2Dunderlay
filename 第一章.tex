	\chapter{绪论}
	\section{研究背景}
	随着移动互联网、VR、AR、物联网等技术的诞生和普及,移动无线通信网络的对频谱资源的需求也变得越来越大。频谱资源十分短缺,同时低频段已经十分拥挤,频谱带宽的分配只能朝着高频段发展。然而,高频段的开发面临着很大的困难,这使得高效利用频谱资源更有意义。传统蜂窝系统以基站为中心的网络结构使得小区的覆盖范围和业务提供方式受到很大限制,阻碍了容量的进一步提升,这使得传统网络满足不了这要的需求。
	因此,能够有效提高频谱使用、提升网络容量、满足本地数据共享需求的 D2D技术被引入进来[1][2]。



	到目前为止,无线通信的发展主要经历了5个阶段,随着里斯本会议5G首个关键标准诞生,TSG,3GPP(3rd Generation Partnership Project),5G大规模试验和商用部署奠定了基础已经快速展开。
	随着移动互联网技术的飞速发展,移动无线网络已经成为我们生活、学习、娱乐不可缺少的必备品,而移动无线通信技术本身也在不断地更新换代。


	移动通信技术发展始于二十世纪七十年代,至今为止,移动通信技术的发展经历了5个阶段,如图\ref{fig:f11}所示
	2017年12月在葡萄牙里斯本召开的3GPP(3rd Generation Partnership Project)第78次会议上,3GPP TSG(Technical Specification Groups)RAN(Radio Access Network)全体会议成功完成了首个可商用部署5G(5th-Generation)NR(New Radio)标准的制定。高通、中兴通讯、华为等业内主流厂商表示,5G NR首发版本完成,5G大规模试验和商用部署将迅速展开。


	第一代移动通信技术(1G,1st Generation)采用的主要是频分多址技术(FDMA,Frequency Division Multiple Access)和模拟技术,用于提供模拟语音的业务,相关标准制定于二十世纪八十年代。受传输带宽限制,只能是一种区域性的移动通信系统。主要有美国的AMPS(Advanced Mobile Phone System),英国的TACS(Total Access Communications System),西德的 C-Netz,北欧的NMT(Nordic Mobile Telephone)等。第一代移动通信有许多的不足之处,如制式太多、容量有限、互不兼容、保密性差、提供不了数据业务、通话质量差和无法提供自动漫游等。


	第二代移动通信技术(2G,2nd Generation)主要采用数字的时分多址(TDMA,time division multiple access)技术和码分多址(CDMA,Code Division Multiple Access)技术。主要提供数字话音业务和低速数据业务。GSM、D-AMPS、PDC(日本数字蜂窝系统)和IS-95CDMA等。它克服了模拟移动通信系统的弱点,大大提高了话音质量、保密性能,并可进行省内、省际自动漫游,但是由于第二代采用不同的制式,标准不统一,用户只能在同一制式覆盖的范围内进行漫游,而且带宽也限制了数据业务的应用,也无法实现高速率的业务如移动的多媒体业务。


