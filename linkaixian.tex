\relax 
\providecommand\hyper@newdestlabel[2]{}
\providecommand\HyperFirstAtBeginDocument{\AtBeginDocument}
\HyperFirstAtBeginDocument{\ifx\hyper@anchor\@undefined
\global\let\oldcontentsline\contentsline
\gdef\contentsline#1#2#3#4{\oldcontentsline{#1}{#2}{#3}}
\global\let\oldnewlabel\newlabel
\gdef\newlabel#1#2{\newlabelxx{#1}#2}
\gdef\newlabelxx#1#2#3#4#5#6{\oldnewlabel{#1}{{#2}{#3}}}
\AtEndDocument{\ifx\hyper@anchor\@undefined
\let\contentsline\oldcontentsline
\let\newlabel\oldnewlabel
\fi}
\fi}
\global\let\hyper@last\relax 
\gdef\HyperFirstAtBeginDocument#1{#1}
\providecommand*\HyPL@Entry[1]{}
\HyPL@Entry{0<</S/D>>}
\HyPL@Entry{4<</S/R>>}
\HyPL@Entry{6<</S/R>>}
\@writefile{toc}{\contentsline {chapter}{摘\hskip 1em\relax 要}{I}{section*.1}}
\@writefile{toc}{\contentsline {chapter}{Abstract}{III}{section*.3}}
\@writefile{toc}{\contentsline {chapter}{插图目录}{IX}{section*.6}}
\@writefile{toc}{\contentsline {chapter}{表格目录}{XI}{section*.8}}
\@writefile{toc}{\contentsline {chapter}{算法目录}{XIII}{section*.10}}
\HyPL@Entry{20<</S/D>>}
\@writefile{toc}{\contentsline {chapter}{术语与符号约定}{1}{section*.12}}
\@writefile{toc}{\contentsline {chapter}{\numberline {第一章\hspace  {0.3em}}绪论}{1}{chapter.1}}
\@writefile{lof}{\addvspace {10\p@ }}
\@writefile{lot}{\addvspace {10\p@ }}
\@writefile{toc}{\contentsline {section}{\numberline {1.1}研究背景}{1}{section.1.1}}
\@writefile{toc}{\contentsline {section}{\numberline {1.2}D2D通信技术}{1}{section.1.2}}
\@writefile{toc}{\contentsline {subsection}{\numberline {1.2.1}D2D通信原理}{1}{subsection.1.2.1}}
\@writefile{toc}{\contentsline {subsection}{\numberline {1.2.2}D2D通信模式}{1}{subsection.1.2.2}}
\@writefile{lof}{\contentsline {figure}{\numberline {1.1}{\ignorespaces \relax }}{2}{figure.caption.13}}
\providecommand*\caption@xref[2]{\@setref\relax\@undefined{#1}}
\newlabel{fig:f11}{{1.1}{2}{\relax }{figure.caption.13}{}}
\@writefile{toc}{\contentsline {subsection}{\numberline {1.2.3}D2D通信研究现状}{2}{subsection.1.2.3}}
\@writefile{toc}{\contentsline {section}{\numberline {1.3}基于图论的资源分配方法}{2}{section.1.3}}
\@writefile{toc}{\contentsline {section}{\numberline {1.4}论文研究的内容及意义}{2}{section.1.4}}
\@writefile{toc}{\contentsline {section}{\numberline {1.5}论文的组织结构}{2}{section.1.5}}
\@writefile{lof}{\contentsline {figure}{\numberline {1.2}{\ignorespaces \relax }}{3}{figure.caption.14}}
\newlabel{fig:f12}{{1.2}{3}{\relax }{figure.caption.14}{}}
\@writefile{toc}{\contentsline {chapter}{\numberline {第二章\hspace  {0.3em}}蜂窝D2D通信系统实时干扰图构建算法}{5}{chapter.2}}
\@writefile{lof}{\addvspace {10\p@ }}
\@writefile{lot}{\addvspace {10\p@ }}
\@writefile{toc}{\contentsline {section}{\numberline {2.1}引言}{5}{section.2.1}}
\@writefile{toc}{\contentsline {section}{\numberline {2.2}系统模型}{5}{section.2.2}}
\@writefile{toc}{\contentsline {subsection}{\numberline {2.2.1}无线资源}{6}{subsection.2.2.1}}
\@writefile{toc}{\contentsline {subsection}{\numberline {2.2.2}小区构成}{6}{subsection.2.2.2}}
\@writefile{toc}{\contentsline {subsection}{\numberline {2.2.3}算法性能描述}{6}{subsection.2.2.3}}
\newlabel{eq2.1}{{2.1}{6}{算法性能描述}{equation.2.2.1}{}}
\@writefile{toc}{\contentsline {section}{\numberline {2.3}干扰图构建算法}{7}{section.2.3}}
\@writefile{toc}{\contentsline {subsection}{\numberline {2.3.1}数据模型}{7}{subsection.2.3.1}}
\@writefile{toc}{\contentsline {subsection}{\numberline {2.3.2}干扰图构建算法的流程}{8}{subsection.2.3.2}}
\newlabel{eq2.1}{{2.2}{10}{干扰图构建算法的流程}{equation.2.3.2}{}}
\newlabel{eq2.1}{{2.3}{10}{干扰图构建算法的流程}{equation.2.3.3}{}}
\newlabel{eq2.1}{{2.4}{11}{干扰图构建算法的流程}{equation.2.3.4}{}}
\newlabel{eq2.1}{{2.5}{11}{干扰图构建算法的流程}{equation.2.3.5}{}}
\newlabel{eq2.1}{{2.6}{11}{干扰图构建算法的流程}{equation.2.3.6}{}}
\newlabel{eq2.1}{{2.7}{11}{干扰图构建算法的流程}{equation.2.3.7}{}}
\newlabel{eq2.1}{{2.8}{11}{干扰图构建算法的流程}{equation.2.3.8}{}}
\newlabel{eq2.1}{{2.9}{12}{干扰图构建算法的流程}{equation.2.3.9}{}}
\newlabel{eq2.1}{{2.10}{12}{干扰图构建算法的流程}{equation.2.3.10}{}}
\newlabel{eq2.1}{{2.11}{12}{干扰图构建算法的流程}{equation.2.3.11}{}}
\newlabel{eq2.1}{{2.12}{13}{干扰图构建算法的流程}{equation.2.3.12}{}}
\newlabel{eq2.1}{{2.13}{13}{干扰图构建算法的流程}{equation.2.3.13}{}}
\newlabel{eq2.1}{{2.14}{13}{干扰图构建算法的流程}{equation.2.3.14}{}}
\newlabel{eq2.1}{{2.15}{13}{干扰图构建算法的流程}{equation.2.3.15}{}}
\newlabel{eq2.1}{{2.16}{13}{干扰图构建算法的流程}{equation.2.3.16}{}}
\newlabel{eq2.1}{{2.17}{13}{干扰图构建算法的流程}{equation.2.3.17}{}}
\newlabel{eq2.1}{{2.18}{14}{干扰图构建算法的流程}{equation.2.3.18}{}}
\newlabel{eq2.1}{{2.19}{14}{干扰图构建算法的流程}{equation.2.3.19}{}}
\newlabel{eq2.1}{{2.20}{14}{干扰图构建算法的流程}{equation.2.3.20}{}}
\newlabel{eq2.1}{{2.21}{14}{干扰图构建算法的流程}{equation.2.3.21}{}}
\newlabel{eq2.1}{{2.22}{14}{干扰图构建算法的流程}{equation.2.3.22}{}}
\newlabel{eq2.1}{{2.23}{14}{干扰图构建算法的流程}{equation.2.3.23}{}}
\newlabel{eq2.1}{{2.24}{14}{干扰图构建算法的流程}{equation.2.3.24}{}}
\newlabel{eq2.1}{{2.25}{15}{干扰图构建算法的流程}{equation.2.3.25}{}}
\newlabel{eq2.1}{{2.26}{15}{干扰图构建算法的流程}{equation.2.3.26}{}}
\newlabel{eq2.1}{{2.27}{15}{干扰图构建算法的流程}{equation.2.3.27}{}}
\newlabel{eq2.1}{{2.28}{15}{干扰图构建算法的流程}{equation.2.3.28}{}}
\newlabel{eq2.1}{{2.29}{15}{干扰图构建算法的流程}{equation.2.3.29}{}}
\@writefile{toc}{\contentsline {subsection}{\numberline {2.3.3}本章小结}{15}{subsection.2.3.3}}
\@writefile{toc}{\contentsline {chapter}{\numberline {第三章\hspace  {0.3em}}干扰图构建算法的理论分析}{17}{chapter.3}}
\@writefile{lof}{\addvspace {10\p@ }}
\@writefile{lot}{\addvspace {10\p@ }}
\@writefile{toc}{\contentsline {section}{\numberline {3.1}引言}{17}{section.3.1}}
\@writefile{toc}{\contentsline {section}{\numberline {3.2}CToD类型干扰边分析}{17}{section.3.2}}
\@writefile{toc}{\contentsline {subsection}{\numberline {3.2.1}蜂窝链路的优先级状态}{17}{subsection.3.2.1}}
\newlabel{eq3.1}{{3.1}{17}{蜂窝链路的优先级状态}{equation.3.2.1}{}}
\newlabel{eq3.1}{{3.2}{17}{蜂窝链路的优先级状态}{equation.3.2.2}{}}
\@writefile{toc}{\contentsline {subsubsection}{场景一:$2*RuNu{m_{Cellular}} \ge Nu{m_{Cellular}}$}{18}{section*.15}}
\@writefile{toc}{\contentsline {subsubsection}{场景二:$Nu{m_{vCellular}} \le 2*RuNu{m_{Cellular}} < Nu{m_{Cellular}}$}{18}{section*.16}}
\newlabel{eq3.1}{{3.3}{18}{场景二:$Nu{m_{vCellular}} \le 2*RuNu{m_{Cellular}} < Nu{m_{Cellular}}$}{equation.3.2.3}{}}
\newlabel{eq3.1}{{3.4}{18}{场景二:$Nu{m_{vCellular}} \le 2*RuNu{m_{Cellular}} < Nu{m_{Cellular}}$}{equation.3.2.4}{}}
\newlabel{eq3.1}{{3.5}{18}{场景二:$Nu{m_{vCellular}} \le 2*RuNu{m_{Cellular}} < Nu{m_{Cellular}}$}{equation.3.2.5}{}}
\newlabel{eq3.1}{{3.6}{18}{场景二:$Nu{m_{vCellular}} \le 2*RuNu{m_{Cellular}} < Nu{m_{Cellular}}$}{equation.3.2.6}{}}
\newlabel{eq3.1}{{3.7}{18}{场景二:$Nu{m_{vCellular}} \le 2*RuNu{m_{Cellular}} < Nu{m_{Cellular}}$}{equation.3.2.7}{}}
\newlabel{eq3.1}{{3.8}{19}{场景二:$Nu{m_{vCellular}} \le 2*RuNu{m_{Cellular}} < Nu{m_{Cellular}}$}{equation.3.2.8}{}}
\newlabel{eq3.1}{{3.9}{19}{场景二:$Nu{m_{vCellular}} \le 2*RuNu{m_{Cellular}} < Nu{m_{Cellular}}$}{equation.3.2.9}{}}
\newlabel{eq3.1}{{3.10}{19}{场景二:$Nu{m_{vCellular}} \le 2*RuNu{m_{Cellular}} < Nu{m_{Cellular}}$}{equation.3.2.10}{}}
\newlabel{eq3.1}{{3.11}{19}{场景二:$Nu{m_{vCellular}} \le 2*RuNu{m_{Cellular}} < Nu{m_{Cellular}}$}{equation.3.2.11}{}}
\newlabel{eq3.1}{{3.12}{19}{场景二:$Nu{m_{vCellular}} \le 2*RuNu{m_{Cellular}} < Nu{m_{Cellular}}$}{equation.3.2.12}{}}
\@writefile{toc}{\contentsline {subsubsection}{场景三:$2*RuNu{m_{Cellular}} \le Nu{m_{vCellular}}$}{19}{section*.17}}
\newlabel{eq3.1}{{3.13}{19}{场景三:$2*RuNu{m_{Cellular}} \le Nu{m_{vCellular}}$}{equation.3.2.13}{}}
\@writefile{toc}{\contentsline {subsection}{\numberline {3.2.2}CToD类型干扰边少边率计算}{20}{subsection.3.2.2}}
\newlabel{eq3.1}{{3.14}{20}{CToD类型干扰边少边率计算}{equation.3.2.14}{}}
\newlabel{eq3.1}{{3.15}{20}{CToD类型干扰边少边率计算}{equation.3.2.15}{}}
\newlabel{eq3.1}{{3.16}{20}{CToD类型干扰边少边率计算}{equation.3.2.16}{}}
\newlabel{eq3.1}{{3.17}{20}{CToD类型干扰边少边率计算}{equation.3.2.17}{}}
\newlabel{eq3.1}{{3.18}{20}{CToD类型干扰边少边率计算}{equation.3.2.18}{}}
\@writefile{toc}{\contentsline {section}{\numberline {3.3}DToD类型干扰边分析}{20}{section.3.3}}
\newlabel{eq3.1}{{3.19}{21}{DToD类型干扰边分析}{equation.3.3.19}{}}
\@writefile{toc}{\contentsline {subsection}{\numberline {3.3.1}循环随机算法分析}{21}{subsection.3.3.1}}
\newlabel{eq3.1}{{3.20}{21}{循环随机算法分析}{equation.3.3.20}{}}
\newlabel{eq3.1}{{3.21}{21}{循环随机算法分析}{equation.3.3.21}{}}
\newlabel{eq3.1}{{3.22}{22}{循环随机算法分析}{equation.3.3.22}{}}
\newlabel{eq3.1}{{3.23}{22}{循环随机算法分析}{equation.3.3.23}{}}
\newlabel{eq3.1}{{3.24}{22}{循环随机算法分析}{equation.3.3.24}{}}
\newlabel{eq3.1}{{3.25}{22}{循环随机算法分析}{equation.3.3.25}{}}
\newlabel{eq3.1}{{3.26}{22}{循环随机算法分析}{equation.3.3.26}{}}
\@writefile{toc}{\contentsline {subsection}{\numberline {3.3.2}最小冲突和算法}{22}{subsection.3.3.2}}
\newlabel{eq3.1}{{3.27}{23}{最小冲突和算法}{equation.3.3.27}{}}
\newlabel{eq3.1}{{3.3.2}{23}{最小冲突和算法}{equation.3.3.28}{}}
\newlabel{eq3.1}{{3.29}{23}{最小冲突和算法}{equation.3.3.29}{}}
\newlabel{eq3.1}{{3.30}{23}{最小冲突和算法}{equation.3.3.30}{}}
\newlabel{eq3.1}{{3.31}{23}{最小冲突和算法}{equation.3.3.31}{}}
\newlabel{eq3.1}{{3.32}{23}{最小冲突和算法}{equation.3.3.32}{}}
\newlabel{eq3.1}{{3.33}{23}{最小冲突和算法}{equation.3.3.33}{}}
\newlabel{eq3.1}{{3.34}{23}{最小冲突和算法}{equation.3.3.34}{}}
\@writefile{toc}{\contentsline {chapter}{\numberline {第四章\hspace  {0.3em}}使用说明}{25}{chapter.4}}
\@writefile{lof}{\addvspace {10\p@ }}
\@writefile{lot}{\addvspace {10\p@ }}
\@writefile{toc}{\contentsline {section}{\numberline {4.1}模板整体框架}{25}{section.4.1}}
\@writefile{toc}{\contentsline {section}{\numberline {4.2}详细说明}{25}{section.4.2}}
\@writefile{toc}{\contentsline {subsection}{\numberline {4.2.1}文档类及其选项}{25}{subsection.4.2.1}}
\@writefile{toc}{\contentsline {subsubsection}{这个不知道怎么样}{25}{section*.18}}
\@writefile{toc}{\contentsline {subsection}{\numberline {4.2.2}载入更多宏包}{26}{subsection.4.2.2}}
\@writefile{toc}{\contentsline {subsection}{\numberline {4.2.3}基本信息设置}{26}{subsection.4.2.3}}
\@writefile{toc}{\contentsline {subsection}{\numberline {4.2.4}中英文摘要}{27}{subsection.4.2.4}}
\@writefile{toc}{\contentsline {subsection}{\numberline {4.2.5}目录与图表目录等}{27}{subsection.4.2.5}}
\@writefile{toc}{\contentsline {subsection}{\numberline {4.2.6}正文章节}{28}{subsection.4.2.6}}
\@writefile{toc}{\contentsline {subsection}{\numberline {4.2.7}致谢}{29}{subsection.4.2.7}}
\@writefile{toc}{\contentsline {subsection}{\numberline {4.2.8}参考文献}{29}{subsection.4.2.8}}
\@writefile{toc}{\contentsline {subsection}{\numberline {4.2.9}附录}{29}{subsection.4.2.9}}
\@writefile{toc}{\contentsline {subsection}{\numberline {4.2.10}作者简介}{29}{subsection.4.2.10}}
\citation{knuth}
\citation{mittlebach}
\@writefile{toc}{\contentsline {chapter}{\numberline {第五章\hspace  {0.3em}}注意事项}{31}{chapter.5}}
\@writefile{lof}{\addvspace {10\p@ }}
\@writefile{lot}{\addvspace {10\p@ }}
\@writefile{toc}{\contentsline {section}{\numberline {5.1}文献引用}{31}{section.5.1}}
\@writefile{toc}{\contentsline {section}{\numberline {5.2}参考文献格式}{31}{section.5.2}}
\@writefile{toc}{\contentsline {section}{\numberline {5.3}图表格式处理}{31}{section.5.3}}
\@writefile{lof}{\contentsline {figure}{\numberline {5.1}{\ignorespaces \makebox  {S\hspace  {-0.3ex}\raisebox  {-0.5ex}{E}\hspace  {-0.3ex}U\hspace  {0.1em}\textsc  {Thesix}} logo\relax }}{32}{figure.caption.19}}
\newlabel{logo}{{5.1}{32}{\seuthesix logo\relax }{figure.caption.19}{}}
\@writefile{toc}{\contentsline {section}{\numberline {5.4}算法格式处理}{32}{section.5.4}}
\newlabel{algoinsight}{{5.1}{32}{如何使用\seuthesix 文档类\relax }{algorithm.5.1}{}}
\@writefile{loa}{\contentsline {algorithm}{\numberline {5.1}{\ignorespaces 如何使用\makebox  {S\hspace  {-0.3ex}\raisebox  {-0.5ex}{E}\hspace  {-0.3ex}U\hspace  {0.1em}\textsc  {Thesix}} 文档类\relax }}{32}{algorithm.5.1}}
\@writefile{toc}{\contentsline {section}{\numberline {5.5}术语生成}{33}{section.5.5}}
\@writefile{toc}{\contentsline {chapter}{\numberline {第六章\hspace  {0.3em}}\texttt  {seuthesix.bst}参考文献格式}{35}{chapter.6}}
\@writefile{lof}{\addvspace {10\p@ }}
\@writefile{lot}{\addvspace {10\p@ }}
\newlabel{bst}{{六}{35}{\texttt {seuthesix.bst}参考文献格式}{chapter.6}{}}
\@writefile{toc}{\contentsline {section}{\numberline {6.1}简介}{35}{section.6.1}}
\@writefile{toc}{\contentsline {section}{\numberline {6.2}支持的Entry type}{35}{section.6.2}}
\citation{komine2004fundamental}
\citation{dimitrov2015principles}
\citation{fujimoto2014fastest}
\citation{ieee2012ieee}
\citation{irdawebsite}
\citation{vlcnews}
\citation{pt}
\citation{thesis:a}
\citation{thesis:b}
\@writefile{lot}{\contentsline {table}{\numberline {6.1}{\ignorespaces 不同entry type支持的field\relax }}{36}{table.caption.20}}
\newlabel{entrytable}{{6.1}{36}{不同entry type支持的field\relax }{table.caption.20}{}}
\@writefile{toc}{\contentsline {section}{\numberline {6.3}各 entry type 所支持的field}{36}{section.6.3}}
\@writefile{toc}{\contentsline {chapter}{\numberline {第七章\hspace  {0.3em}}全文总结}{37}{chapter.7}}
\@writefile{lof}{\addvspace {10\p@ }}
\@writefile{lot}{\addvspace {10\p@ }}
\bibstyle{seuthesix}
\@writefile{toc}{\contentsline {chapter}{致 谢}{39}{section*.21}}
\bibdata{seuthesix}
\@writefile{toc}{\contentsline {chapter}{参考文献}{41}{section*.23}}
\@writefile{toc}{\contentsline {chapter}{作者简介}{41}{section*.24}}
